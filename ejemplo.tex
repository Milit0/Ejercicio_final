\documentclass[a4paper,11pt]{article}

\usepackage[utf8]{inputenc}
\usepackage{color}
\usepackage{array}
\usepackage{amsmath,amssymb}
\usepackage{graphicx}

\addtolength{\textwidth}{2cm}
\addtolength{\hoffset}{-1cm}

\title{Avances informáticos}
\author{Emilio Gonzalez}

\begin{document}

\maketitle



\section{Introducción}

En el presente trabajo presentamos una revisión rápida sobre los últimos avances  en los formatos de comunicación a través de Internet y su nueva conformación como medio de comunicación global.


\section{Desarrollo}

Hasta hace unos pocos años, los servicios de Internet tuvieron un crecimiento dinámico. Sin embargo, lo servicios eran relativamente estáticos: Una persona creaba un blog, y sus amigos y otras personas, leían con interés lo que publicaba. Una institución cultural o gubernamental publicaba la información, la cual se actualizaba siguiendo las sugerencias de los usuarios, pero dentro de un esquema estático y jerárquico, similar a los medios de comunicación tradicionales, tales como la radio y la televisión, donde es el difusor del medio quien decide qué se publica y qué no. Esto es el concepto de la WEB 1.0. La generalización de las redes sociales, como Facebook y Twitter, las posibilidades de comentar publicaciones y la libertad de los usuarios para elegir y corregir contenidos (como wikipedia), han cambiado la perspectiva estática de la comunicación en Internet, haciéndola un medio más social y de mayor interacción entre creadores de contenidos y usuarios. Esto es la WEB 2.0


\begin{figure}[h]
\begin{center} 
\bigskip
\includegraphics[trim=15 30 30 0,clip,height=0.3\textwidth]{facebook.jpg} \hspace{5mm}

\end{center}
\vspace{-5mm}
\caption{Facebook}
\end{figure}

\section{Desarrollo} 

En la actualidad, estamos al final de esta interacción entre usuarios  y creadores de contenido, ya que esta relación está evolucionando. La creación de aplicaciones inteligentes que predicen las preferencias de los usuarios, sistemas operativos con mayor interactividad y dispositivos móviles que prácticamente permiten una conexión permanente,  comienzan a configurar el siguiente peldaño en esta evolución comunicativa la  WEB 3.0.






\begin{figure}[t]
\begin{center} 
\bigskip
\includegraphics[height=0.25\textwidth]{twitter.png} \hspace{5mm}
\begin{tabular}[b]{|c|c|c|}
\hline
A & B & C  \\ \hline
Uno & Dos & Tres \\
Cuatro & Cinco & Seis \\ \hline
\end{tabular}
\end{center}
\vspace{-5mm}
\caption{Twitter}
\label{tux}
\end{figure}

\section{Conclusión}
Estamos en un punto intermedio en la evolución de Internet. La conectividad permanente para muchos ya es una realidad, y la reducción de  costos en los servicios, así como el considerar Internet como un derecho humano, acercan la idea de la WEB 3.0 a una realidad.





\addcontentsline{toc}{section}{References}
\begin{thebibliography}{99}
\bibitem{unistan} www.ejemplode.com
\bibitem{acp} \begin{equation}
\Phi = \oint_s \vec E \cdot d\vec S = 
\frac{q_{enc}}{\varepsilon_0} \quad \text{(Ley de Gauss)}
\end{equation}

\end{thebibliography}


\end{document}